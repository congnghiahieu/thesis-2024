\section{Ứng dụng bài toán học máy trên đồ thị CPG}

% Mục tiêu thực nghiệm là Bài toán phân loại
% Dữ liệu
% Độ đo đánh giá
% Baseline
% Kết quả thực nghiệm

Để chứng minh tiềm năng của đồ thị CPG dành cho ngôn ngữ Rust, khóa luận xây dựng một thực nghiệm sử dụng kĩ thuật học máy để phát hiện lỗ hổng bảo mật trong mã nguồn Rust dựa trên đồ thị CPG.
Bài toán đặt ra là phân loại tệp mã nguồn Rust có lỗ hổng bảo mật và không có lỗ hổng bảo mật.
Thực nghiệm sẽ tiến hành so sánh độ hiệu quả giữa mô hình học máy sử dụng đồ thị CPG và mô hình học máy học sử dụng mã nguồn Rust truyền thống.
Mô hình học máy chỉ sử dụng dữ liệu mã nguồn Rust, tạm gọi là Baseline, sẽ sử dụng kĩ thuật Word2Vec \cite{church2017word2vec} để biểu diễn mã nguồn và Logistic Regression \cite{lavalley2008logistic} để phân loại.
Đối với bài toán học máy để phát hiện lỗ hổng bảo mật trên đồ thị CPG, trước đây đã có phương pháp Devign, một phương pháp kinh điển cho lớp bài toán này \cite{zhou2019devign}.
Devign đã sử dụng đồ thị CPG cho ngôn ngữ C/C++, do đó hoàn toàn có thể áp dụng phương pháp Devign với đồ thị CPG cho ngôn ngữ Rust trong khóa luận này.

Bộ dữ liệu được sử dụng bao gồm 277 tệp mã nguồn, trong đó 138 tệp có lỗ hổng bảo mật và 139 tệp không có lỗ hổng bảo mật.
Đối với 138 tệp có lỗ hổng bảo mật, các tệp này được trích xuất từ bộ dữ liệu trong nghiên cứu của David Lo và cộng sự \cite{zheng2023closer}.
Còn 139 tệp không có lỗ bảo mật được thu thập từ 100 dự án mã nguồn mở của Rust nhiều sao nhất trên Github \cite{githubGithubRankingTop100RustmdMaster} và xác nhận thủ công.
Tập dữ liệu được chia thành 3 phần, 80\% dữ liệu được sử dụng cho quá trình huấn luyện, 10\% dữ liệu được sử dụng cho kiểm chứng, 10\% dữ liệu được sử dụng cho kiểm thử.
Để đánh giá, mỗi phương pháp được chạy 10 lần đối với tập dữ liệu như trên.
Kết quả của các chỉ số là giá trị trung bình của 10 lần chạy.
Kết quả thực nghiệm được thể hiện bằng bảng phía dưới.

Với bài toán phân loại, thang đo đánh giá hiệu suất của phương pháp sẽ bao gồm các chỉ số Accuracy, Precision, Recall và F1-score.
% ROC AUC, Precision-Recall AUC, MCC, Error Rate
Accuracy biểu thị tỷ lệ phần trăm các dự đoán đúng trên tổng số các dự đoán, thể hiện độ chính xác tổng thể của mô hình, nhưng có thể không phản ánh đúng với dữ liệu mất cân bằng.
Precision đo độ chính xác trong việc phân loại đúng một lớp cụ thể, hữu ích khi cần giảm thiểu dự đoán dương tính sai.
Recall, hay độ nhạy, đánh giá khả năng phát hiện đầy đủ các mẫu thuộc một lớp, đặc biệt quan trọng trong việc giảm thiểu bỏ sót các trường hợp dương tính.
F1-score là chỉ số thể hiện sự cân bằng giữa Precision và Recall, giúp đánh giá hiệu suất chung khi cả hai chỉ số này đều quan trọng.


\begin{table}[H]
    \centering
    \caption{Bảng so sánh phương pháp Baseline và phương pháp Devign (Rust).}
    \label{table:c4_ml}
    \begin{tabular}{l @{\hskip 3cm} c @{\hskip 3cm} c}
        \hline
         & Baseline & \textbf{Devign (Rust)} \\
        \hline
        Accuracy & 0.73 & \textbf{0.77} \\
        Precision & 0.73 & \textbf{0.77} \\
        Recall & 0.70 & \textbf{0.99} \\
        F1-score & 0.72 & \textbf{0.87} \\
        % ROC AUC & 0.81 & \textbf{0.47} \\
        % Precision-Recall AUC & 0.74 & \textbf{0.78} \\
        % MCC & 0.46 & \textbf{0.01} \\
        % Error Rate & 0.27 & \textbf{0.23} \\
        \hline
    \end{tabular}
\end{table}

Trong Bảng \ref{table:c4_ml}, phương pháp Devign đã cho kết quả tốt hơn so với phương pháp Baseline với hầu hết các chỉ số.
Ta thấy được Accuracy hơn 0.04, Precision hơn 0.04, Recall hơn 0.29, F1-Score hơn 0.15.
% , Precision-Recall AUC hơn 0.04 và Error Rate thấp hơn 0.04.
% Tuy nhiên, ROC AUC và MCC của phương pháp Devign lại thấp hơn so với phương pháp Baseline.
Điều này chứng tỏ rằng đồ thị CPG cho ngôn ngữ Rust có tiềm năng áp dụng cho bài toán phát hiện lỗ hổng bảo mật kết hợp bằng kĩ thuật học máy.
Mã nguồn và bộ dữ liệu sử dụng trong thực nghiệm được lưu trữ lần lượt tại địa chỉ \href{https://github.com/congnghiahieu/devign}{devign}, \href{https://github.com/congnghiahieu/rust-ecosystem}{rust-ecosystem}.

Độ chính xác của phương pháp Devign cho ngôn ngữ Rust mới chỉ đạt được ở mức 0.77 bởi một vài nguyên do.
Thứ nhất là kích cỡ của bộ dữ liệu cho ngôn ngữ Rust.
Hiện tại các mã nguồn có lỗi được xác nhận đều lấy từ RUSTSEC Database.
Rust là ngôn ngữ mới phát triển gần đây, hệ sinh thái chưa lớn mạnh nên số lượng mã nguồn có lỗi được báo cáo không nhiều.
Thứ hai là đồ thị CPG của Rust vẫn chưa đầy đủ hoàn toàn.
Tồn tại các tính năng của Rust như marco, module chưa thể phân tích ra cây AST hay lấy được các lớp thông tin cần thiết.
Các hạn chế này sẽ được trình bày chi tiết ở phần \ref{sec:limit}.
% Các lớp thông tin về CFG, PDG vẫn chưa được hoàn thiện, hay các lớp thông tin khác có giá trị khai thác được sinh ra từ Joern cũng chưa được cung cấp.
Thứ ba là giới hạn trong cài đặt của phương pháp Devign.
Hiện tại, phiên bản cài đặt của Devign là mô phỏng lại từ bài báo gốc.
Devign mới chỉ có thể sử dụng lớp thông tin về AST và CFG mà không có PDG, do vậy không thể khai thác được hết toàn bộ thông tin từ đồ thị CPG.
% chưa kể các lớp thông tin khác của Joern CPG.
Nếu có thể khắc phục được các hạn chế kể trên, phương pháp Devign có thể đạt được kết quả tốt hơn nữa.

Dù tồn tại những hạn chế, kết quả thực nghiệm vẫn cho thấy tiềm năng khai thác to lớn của đồ thị CPG cho ngôn ngữ Rust.
Đồ thị CPG có thể được sử dụng cho các lớp bài toán cần đến phân tích mã nguồn như phát hiện lỗ hổng bảo mật, phân loại mã nguồn hay các ứng dụng khác trong tương lai.
Ngoài ra, việc cải thiện và hoàn thiện đồ thị CPG cho ngôn ngữ Rust sẽ mở ra nhiều hướng nghiên cứu và ứng dụng mới.
Điều này không chỉ giúp nâng cao hiệu quả của các phương pháp hiện tại mà còn thúc đẩy sự phát triển của hệ sinh thái ngôn ngữ Rust.
