\section{Hạn chế}
\label{sec:limit}

\subsection{Chưa hỗ trợ cú pháp Macro}

Trong ngôn ngữ lập trình C/C++ và Rust có khái niệm macro, là cách viết mã nguồn để sinh ra mã nguồn khác \cite{rustlangMacrosRust}.
% Macro trong Rust bao gồm 2 loại declarative macro và procedural macro.
% Declarative macro giống như macro trong C/C++ \cite{}, còn procedural macro giống như inline function.
Ngôn ngữ C/C++ có bộ tiền xử lý trước khi cho mã nguồn vào trình biên dịch, do đó sau khi được tiền xử lý thì mã nguồn C/C++ đã được mở rộng toàn bộ macro.
Tuy nhiên, Rust không giống C/C++, macro của Rust không được xử lý trước khi sinh cây AST mà sẽ được xử lý sau khi sinh cây AST nhưng trước khi đi vào pha phân tích ngữ cảnh của trình biên dịch \cite{veykrilSourceAnalysis}.
Như đã đề cập ở phần các trước, công cụ hiện tại sử dụng thư viện syn để sinh cây AST cho mã nguồn và syn không hỗ trợ xử lý macro.
Do đó tất cả các mã lệnh nằm bên trong macro sẽ không được xử lý, dẫn đến việc không thể sinh cây AST cho đoạn mã lệnh sử dụng macro.
Tất cả các đoạn lệnh nằm trong một lời gọi macro hiện tại được xem như một chuỗi ký tự.
Không chỉ vậy macro trong Rust sử dụng DSL (Domain-Specific Language) riêng.
DSL này gần với ngôn ngữ Rust nhưng có sự mở rộng, biến đổi để phù hợp với tính năng macro, do đó gây khó khăn cho việc sinh cây AST.

Để xử lý trường hợp trên, một bước tiền xử lý mã nguồn có thể được thêm vào để mở rộng macro như C/C++.
Công cụ có thể sử dụng đến tính năng mở rộng marco của trình biên dịch Rust \cite{rustlangMacroExpansion}.
Tính năng này có tác dụng đưa đoạn mã macro mà lập trình viên nhìn thấy thành đoạn mã mà trình biên dịch nhìn thấy.
Mã nguồn sau khi được mở rộng sẽ thay thế các lời gọi macro bằng định nghĩa gốc của macro đó.
Các macro được nạp sẵn của ngôn ngữ Rust như \texttt{println!}, \texttt{vec!} hay kể cả macro do người dùng định nghĩa cũng sẽ được xử lý.
Không chỉ vậy, mã nguồn sau khi được mở rộng sẽ có được các thông tin bị ẩn đi như các mặc định của Rust bao gồm các hàm, các biến được nạp sẵn trong ngôn ngữ.
Tuy nhiên, việc mở rộng macro trước khi sinh cây AST sẽ làm cho mã nguồn bị biến đổi so với mã nguồn gốc, đồng thời tăng kích cỡ và độ lớn của mã nguồn.
Việc thêm các thông tin ẩn mà lập trình viên không nhìn thấy ở mã nguồn gốc có thể gây nhầm lẫn.
Điều này cũng đồng nghĩa với việc quá trình sinh cây AST cho mã nguồn sau khi xử lý macro sẽ phức tạp và tốn nhiều thời gian hơn.

\subsection{Chưa hỗ trợ cơ chế Module}

Cơ chế module trong Rust tương tự namespace trong C++ hay package trong Java \cite{rustlangManagingGrowing}.
Module chia nhỏ mã nguồn thành các phần nhỏ hơn, tổ chức, sắp xếp và quản lý mã nguồn.
Module giúp tái sử dụng mã nguồn, tránh xung đột tên biến, tên hàm giữa các module khác nhau.
Mỗi module trong Rust có thể là một tệp riêng biệt hoặc nằm chung trong một tệp mã nguồn.
Các module có thể được tổ chức thành một hệ thống phân cấp, với các module con được khai báo bên trong các module cha.
Ngoài ra, có một số tệp được coi là tệp đặc biệt trong cấu trúc mã nguồn Rust.
Ví dụ như tệp với tên \texttt{mod.rs}, đây là tệp module gốc của thư mục chứa nó.
Tất cả các module con trong thư mục đó sẽ không được phép truy cập trực tiếp từ bên ngoài mà phải thông qua tệp \texttt{mod.rs}.
Còn có tệp tên \texttt{main.rs} hoặc \texttt{lib.rs} để đánh dấu điểm đầu vào của chương trình và xác định xem dự án là một thư viện hay một ứng dụng.
Rust còn cung cấp cơ chế workspace, cho phép quản lý nhiều dự án nhỏ trong cùng một dự án lớn, mỗi dự án là một thư mục con trong thư mục gốc lớn nhất \cite{rustlangCargoWorkspaces}.
Để kiểm soát khả năng truy cập, Rust sử dụng cơ chế visibility \cite{rustlangVisibilityPrivacy}.
Mặc định các thành phần trong module chỉ có phạm vi truy cập giới hạn trong module nó được định nghĩa.
Để làm cho một đơn vị có thể truy cập được từ các module khác, ta sử dụng từ khóa \texttt{pub} để đánh dấu cho đơn vị cần mở rộng quyền truy cập.
Rust còn có cơ chế đường dẫn để định danh một thành phần trong mã nguồn.
Ta có thể truy cập đến thành phần của module khác bằng cách chỉ ra đường dẫn định danh hợp lệ.
Một đường dẫn có thể là đường dẫn tuyệt đối hoặc đường dẫn tương đối.
Ví dụ từ khóa \texttt{self} dùng để chỉ tới module hiện tại, từ khóa \texttt{super} để chỉ tới module cha của module hiện tại, từ khóa \texttt{crate} để chỉ tới module gốc của dự án.
Hệ thống module phức tạp của Rust làm tăng đáng kể độ khó cho việc xử lý quan hệ giữa các module trong cây AST và phân tích ngữ cảnh.
Các khái niệm như workspace, module, visibility, cơ chế đường dẫn và những tệp tin đặc biệt tạo nên sự phức tạp cho việc phân tích ngữ cảnh của mã nguồn.
Hiện tại công cụ chưa xử lý được các tính năng xung quanh các khái niệm này.

% \subsection{Path}

% \begin{itemize}
%     \item Path có thể là Absolute Path hoặc Relative Path tùy vào bối cảnh module hiện tại, path có thể chỉ tới đối tượng trong cùng một module hoặc khác module
%     \item Path trong Rust là đường dẫn đến một đối tượng nào đó được định nghĩa trong mã nguồn như struct, trait, static, const, function
%     \item Cây AST sử dụng thư viện $syn$, $syn$ có thể lấy được path của một đối tượng nhưng không biết được đối đượng đang trỏ tới là static, const, hay function.
% Do đó đang không phân biệt được đâu là path của static, const, hay function
% \end{itemize}

% \subsection{Type Argument match Type Parameter}

% \subsection{Các lớp thông tin khác của đồ thị CPG}

% Như đã trình bày ở phần kiến trúc của công cụ Joern, một đồ thị CPG sẽ bao gồm nhiều lớp thông tin khác nhau như AST, CFG, PDG hay các lớp thông tin khác của Joern và đặc thù của ngôn ngữ.
% Hiện tại công cụ đã xử lý toàn bộ thông tin của lớp AST, phần lớn các thônng tin của lớp CFG, PDG.
% Tuy nhiên vẫn còn một số lớp thông tin khác phục vụ cho việc duyệt đồ thị, liên kết đồ thị như \texttt{tệpSystem}, \texttt{Shortcuts}, \texttt{TagsAndLocation}, \texttt{Annotation} vẫn chưa được đầy đủ.
% Do vậy, một trong các mục tiêu tương lai của công cụ là hoàn thiện các lớp thông tin này.