\chapter{Cài đặt công cụ và thực nghiệm}
\label{chap:experiment}

Chương này sẽ tập trung trình bày về quá trình thực nghiệm và đánh giá phương pháp đánh giá điểm thưởng dựa trên thiết kế đã mô tả chi tiết trong phần trước. Phần thực nghiệm sẽ đi sâu vào việc áp dụng phương pháp để cải thiện hiệu quả công cụ ARAT-RL và so sánh hiệu suất của phương pháp mới với phương pháp đánh giá điểm thưởng mặc định của ARAT-RL. Dựa trên kết quả thu được, tôi sẽ đưa ra các nhận xét và kết luận về giải pháp đề xuất.

\section{Cài đặt công cụ}

Công cụ phân tích mã nguồn Go được phát triển từ mã nguồn của công cụ Joern.
Công cụ được cài đặt bằng ngôn ngữ Go và Scala. Kiến trúc tổng quan của công cụ được
mô tả ở Hình 4.1

Công cụ gồm hai mô-đun là GoParser và JoernParse trong đó JoernParse là mô-đun
chính. Mô-đun GoParse được cài đặt bằng ngôn ngữ Go. Mô-đun này sẽ nhận đầu vào là
đường dẫn đến thư mục dự án chứa mã nguồn Go và thực hiện việc phân tích mã nguồn
thành cây cú pháp mã nguồn, xử lý kiểu dữ liệu và lưu cây cú pháp mã nguồn vào các
tệp JSON. JoernParse được cài đặt bằng ngôn ngữ Scala và được phát triển từ mã nguồn
của công cụ Joern. Mô-đun thực hiện nhận mã nguồn đầu vào, gọi mô-đun GoParser để
dựng cây cú pháp mã nguồn và sau đó xây dựng đồ thị thuộc tính mã nguồn từ các thông
tin trong cây cú pháp mã nguồn. Đầu ra của mô-đun này cũng là đồ thị thuộc tính mã
nguồn được lưu dưới dạng tệp nhị phân (.bin). Đây cũng là đầu ra của cả công cụ. Chúng
ta có thể dùng một số công cụ được Joern cung cấp sẵn để thao tác với đồ thị này như xuất đồ thị dưới nhiều định dạng khác nhau như neo4jcsv, graphml, graphson, dot bằng
công cụ JoernExport, thực hiện các câu lệnh truy vấn trên đồ thị hoặc quét đồ thị để tìm
lỗ hổng trong mã nguồn bằng công cụ JoernScan.

\subsection{Mô-đun xây dựng cây cú pháp trừu tượng Rust Parser}

Chức năng chính của GoParser là nhận thông tin đường dẫn đến mã nguồn, lọc các
tệp mã nguồn Go, sinh cây cú pháp trừu tượng, xử lý kiểu dữ liệu cho các nút định danh
(identifier) và lưu kết quả vào các tệp JSON. Các thành phần trong mô-đun này bao gồm
cli, parser, resolver, ast và util. Chức năng chi tiết của từng thành phần được mô tả như
sau:

\begin{itemize}
\item Thành phần cli làm nhiệm vụ xử lý và cung cấp giao diện dòng lệnh (CLI), nhận
đầu vào là đường dẫn đến dự án Go, nhận các cấu hình (ví dụ như đường dẫn lưu
các tệp đầu ra hoặc danh sách các tệp mã nguồn cần bỏ qua) và gọi các chức năng
phân tích.
\item Thành phần parser thực hiện nhận thông tin từ thành phần cli, từ đó đọc các tệp mã
nguồn của Go có đuôi là ".go" từ dự án đầu vào, phân tích các tệp tin này và sinh
cây cú pháp trừu tượng.
\item Thành phần resolver cung cấp các API phục vụ việc xử lý kiểu dữ liệu.
\item Thành phần ast chứa các cấu trúc định nghĩa các loại nút và thuộc tính của từng
loại nút của cây cú pháp trừu tượng.
\item Thành phần util cung cấp các tiện ích chung bao gồm xử lý tệp tin và xử lý nhật ký
(log).
\end{itemize}
Hình 4.2 biểu diễn các thành phần của mô-đun GoParser, các mũi tên biểu mối
quan hệ sử dụng giữa các thành phần này.

\subsection{Mô-đun xây dựng đồ thị thuộc tính mã nguồn Joern Rust}

Mô-đun JoernParse thực hiện nhiệm vụ xây dựng đồ thị thuộc tính mã nguồn (CPG)
từ cây cú pháp trừu tượng AST. Mô-đun sẽ thực hiện đọc thông tin từ các tệp JSON được
tạo ra từ mô-đun GoParser và xây dựng đồ thị thuộc tính mã nguồn (CPG). Mô-đun này
được phát triển từ mô-đun JoernParse của công cụ Joern nên sử dụng lại một số thành
phần có sẵn của Joern bao gồm các thành phần console, semanticcpg, codepropertygraph
và x2cpg. Dưới đây tôi sẽ mô tả chức năng của từng thành phần:
\begin{itemize}
\item Thành phần console thực hiện chức năng tương tự như thành phần cli của mô-đun
GoParser.
\item Thành phần go-language-frontend đảm nhận nhiệm vụ nhận đường dẫn đến mã nguồn từ thành phần console, gọi mô-đun GoParser để sinh cây cú pháp trừu tượng sau đó đọc các tệp JSON được sinh ra và xây dựng thành đồ thị thuộc tính mã nguồn.
\item Thành phần codepropertygraph chứa các lớp định nghĩa các đỉnh và cạnh của đồ thị thuộc tính mã nguồn.
\item Thành phần x2cpg cung cấp các tiện ích để ánh xạ các thuộc tính từ các nút của cây cú pháp trừu tượng sang thuộc tính của các đỉnh trong đồ thị thuộc tính mã nguồn.
\item Thành phần semanticcpg chứa các tiện ích cho phép vấn đồ thị thuộc tính mã nguồn. Thành phần này được sử để thực hiện các tác vụ hậu xử lý sau khi đồ thị thuộc tính mã nguồn được xây dựng (ví dụ như bổ sung thông tin hoặc tạo các đỉnh liên kết).
\end{itemize}

Hình 4.3 biểu diễn các thành phần của mô-đun GoParser, các mũi tên biểu mối
quan hệ sử dụng giữa các thành phần này.

\section{Thực nghiệm trên các tính năng đã hỗ trợ}

\subsection{Tính năng 1}

\subsection{Tính năng 2}

\subsection{Tính năng 3}

\section{Nhược điểm}

\subsection{Nhược điểm 1}

\subsection{Nhược điểm 2}

\subsection{Nhược điểm 3}