% \chapter{Ứng dụng phân tích mã nguồn Rust bằng đồ thị CPG}
\chapter{Ứng dụng thực nghiệm và đánh giá}
\label{chap:experiment}

% Chương 5 trình bày về cách ánh xạ cây AST sang đồ thị CPG trên các thể loại cú pháp Rust.
% Mức độ cài đặt và các hạn chế hiện thời của công cụ sẽ được thảo luận.
% Không chỉ vậy, công cụ sẽ được sử dụng để phân tích mã nguồn của một số đoạn mã có lỗ hổng bảo mật được công bố trên RUSTSEC Database \cite{rustsecAboutRustSec}.
% Ngoài ra, chương cũng sẽ thực nghiệm ứng dụng đồ thị CPG trong bài toán học máy phân loại mã nguồn Rust có lỗ hổng bảo mật, thực hiện so sánh với mô hình học máy khác và chứng minh tiềm năng khai thác của đồ thị CPG dành cho ngôn ngữ Rust.

Chương 4 trình bày về các ứng dụng phân tích mã nguồn Rust bằng đồ thị CPG bao gồm phân tích mã nguồn có lỗ hổng và kỹ thuật học máy.
Công cụ được sử dụng để phân tích mã nguồn của các đoạn mã có lỗ hổng bảo mật được công bố trên RUSTSEC Database \cite{rustsecAboutRustSec}.
Chương này cũng sẽ ứng dụng đồ thị CPG cho bài toán học máy để phân loại mã nguồn Rust có lỗ hổng bảo mật, thực hiện so sánh với mô hình học máy khác và chứng minh tiềm năng của đồ thị CPG dành cho ngôn ngữ Rust.
Cuối cùng sẽ thảo luận các hạn chế hiện thời của công cụ.
% Ngoài ra, chương cũng sẽ trình bày về các hạn chế hiện thời, nhằm làm rõ các thách thức và định hướng cải tiến công cụ trong tương lai.

\input{chapters/c4/c4_rustsec.tex}
\section{Thực nghiệm Machine Learning trên CPG}

\begin{itemize}
    \item Mục tiêu thực nghiệm là Bài toán phân loại
    \item Dữ liệu
    \item Độ đo đánh giá
    % \item Thực nghiệm được tiến hành trên máy tính với các thông số như sau: Ubuntu 18.04, Intel® CoreTM i5-3470 CPU @ 3.20GHz × 4, 8GBs RAM memory.
    % \item Các thử nghiệm được thực hiện trên máy tính có cấu hình chip xử lý CPU AMD® Ryzen 5 3500u @ 3.1GHz x 4, bộ nhớ RAM 20GB và ổ cứng SSD 512GB
    \item Baseline
    \item Kết quả thực nghiệm
\end{itemize}

Để tăng tính thuyết phục và chứng minh tiềm năng áp dụng được vào thực tiễn của đồ thị CPG dành cho ngôn ngữ Rust mà khóa luận đã thực hiện.
Khóa luận đã thực hiện xây dựng một thực nghiệm bài toán sử dụng học máy để phát hiện lỗ hổng bảo mật cho mã nguồn Rust dựa trên đồ thị CPG.

Thực nghiệm được thực hiện trên máy tính cá nhân với cấu hình CPU AMD Ryzen 7 8845H, RAM 32GB.
Thực hiện chạy 10 lần đối với mỗi mô hình học máy để đánh giá hiệu suất của mô hình.
Kết quả là giá trị trung bình của 10 lần chạy.

Bộ dữ liệu được sử dụng trong thực nghiệm 130 tệp mã nguồn Rust có lỗ hổng bảo mật và 130 tệp mã nguồn Rust không có lỗ hổng bảo mật.
Đối với 130 tệp mã nguồn Rust có lỗ hổng bảo mật được trích xuất từ bộ dữ liệu trong nghiên cứu của David Lo và cộng sự \cite{zheng2023closer}.
130 tệp mã nguồn Rust không có lỗ bảo mật được thu thập từ 100 dự án mã nguồn mở của Rust nhiều sao nhất trên Github \cite{githubGithubRankingTop100RustmdMaster} và thực kiện kiểm chửng thủ công.

Đối với bài toán học máy để phát hiện lỗ hổng bảo mật trên nền tảng CPG thì trước đây đã có phương pháp Devign, một mô hình kinh điển cho lớp bài toán này \cite{zhou2019devign}.
Devign đã được sử dụng cho ngôn ngữ C/C++, cũng sử dụng CPG của Joern, do đó Devign hoàn toàn có áp dụng tương tự cho CPG Rust trong khóa luận này
Đối thủ so sánh của Devign là mô hình học máy sử dụng Word2Vec và Logistic Regression, mô hình này được đặt tên là Baseline.

Do bài toán học máy ở đây rơi vào dạng phân loại nhị phân, nên các thang đo được sử dụng để đánh giá hiệu suất của mô hình là Accuracy, Precision, Recall, F1-score, ROC AUC, Precision-Recall AUC, MCC, Error Rate

Mã nguồn của việc thực nghiệm được lưu trữ tại địa chỉ \href{TODO}{TODO}.
Kết quả thực nghiệm được thể hiện trong Bảng \ref{table:c4_ml}.

\begin{table}[H]
    \centering
    \caption{Bảng so sánh mô hình Baseline và phương pháp Devign}
    \label{table:c4_ml}
    \begin{tabular}{l @{\hskip 2cm} c @{\hskip 2cm} c}
        \hline
         & Baseline & Devign \\
        \hline
        Accuracy & 0.73 & \textbf{0.77} \\
        Precision & 0.73 & \textbf{0.77} \\
        Recall & 0.70 & \textbf{0.99} \\
        F1-score & 0.72 & \textbf{0.87} \\
        ROC AUC & 0.81 & \textbf{0.47} \\
        Precision-Recall AUC & 0.74 & \textbf{0.78} \\
        MCC & 0.46 & \textbf{0.01} \\
        Error Rate & 0.27 & \textbf{0.23} \\
        \hline
    \end{tabular}
\end{table}

Trong Bảng \ref{table:c4_ml}, phương pháp Devign đã cho kết quả tốt hơn so với mô hình Baseline với hầu hết các thang đo đánh giá hiệu suất.
Ta thấy được Accuracy hơn 0.04, Precision hơn 0.04, Recall hơn 0.29, F1-Score hơn 0.15, Precision-Recall AUC hơn 0.04 và Error Rate thấp hơn 0.04.
Tuy nhiên, ROC AUC và MCC của phương pháp Devign lại thấp hơn so với mô hình Baseline.
Điều này chứng tỏ phương pháp Devign có tiềm năng áp dụng cho bài toán học máy phát hiện lỗ hổng bảo mật trên nền tảng CPG Rust, nhưng vẫn cần cải thiện thêm.

Vốn dĩ độ chỉ số Accuracy của phương pháp Devign chỉ đạt được ở mức 0.77 bời nhiều lý do.
Thứ nhất là hạn chế về bộ dữ liệu của ngôn ngữ Rust, hiện tại các mã nguồn có lỗi được xác nhận chủ yếu lấy từ RUSTSEC Database.
Rust là ngôn ngữ mới phát triển gần đây, hệ sinh thái chưa lớn mạnh nên số lượng mã nguồn có lỗi được báo cáo chưa nhiều.
Thứ hai là hạn chế về đồ thị CPG, hiện tại đồ thị CPG của Rust vẫn còn nhiều thiếu sót.
Các lớp thông tin về CFG, PDG vẫn chưa được hoàn thiện, hay các lớp thông tin khác có giá trị khai thác được sinh ra từ Joern cũng được có được cung cấp.
Thứ ba là hạn chế về phiên bản cài đặt của phương pháp Devign.
Phiên bản cài đặt của Devign hiện tại là mô phỏng lại từ bài báo gốc, mã nguồn đã lâu không được bảo trì và phát triển.
Hiện tại Devign chỉ có thể sử dụng lớp thông tin về AST, CFG mà không có PDG, do vậy không thể khai thác được hết toàn bộ thông tin từ đồ thị CPG, chưa kể các lớp thông tin khác của Joern CPG.

Các điểm cần cải thiện của khóa luận bao gồm: hoàn thiện các lớp thông tin cho đồ thị CPG, cải thiện bộ dữ liệu cho ngôn ngữ Rust, phát triển tiếp mô hình học máy Devign cho bài toán phân loại mã nguồn bị lỗi, hay thử nghiệm trên các bài toán học máy ở chủ đề khác.
Dù tồn tại những hạn chế như trên, không thể phủ nhận tiềm năng khai thác của đồ thị CPG cho ngôn ngữ Rust cho các lớp bài toán cần đến phân tích mã nguồn như phát hiện lỗ hổng bảo mật, phân loại mã nguồn, v.v.
\section{Hạn chế}

\subsection{Macro}

Trong các ngôn ngữ lập trình như C/C++ và Rust có macro hay còn gọi là metaprogramming. Có thể hiểu macro (metaprogramming) là viết code để sinh ra code, tức là viết code Rust để sinh ra các đoạn code Rust còn lại. Macro trong Rust có 2 loại: declarative macro và procedural macro. Declarative macro giống như macro trong C/C++, còn procedural macro giống như inline function, được xử lý ở bước sinh cây AST. Đối C/C++ có preprossor để xử lý Marco trước khi cho vào compiler, đo đó khi mã nguồn sau khi được tiền xử lý thì đã được xử lý toàn bộ Macro. Tuy nhiên, Rust không giống C/C++, Marco của Rust \cite{rustlangMacrosRust} không được xử lý ở bước sinh cây AST mà sẽ được xử lý sau khi sinh cây AST nhưng trước khi đi vào phase Semantic Ananlysis của trình biên dịch. Như đã để cập ở trên, công cụ hiện tại sử dụng thư viện syn để sinh cây AST cho mã nguồn Rust và syn không hỗ trợ xử lý Macro. Do đó tất cả các mã lệnh nằm bên trong macro sẽ không được xử lý, dẫn đến việc không thể sinh cây AST cho đoạn mã lệnh sử dụng macro. Tất cả các đoạn lệnh nằm trong 1 lời gọi macro hiện tại được xem như 1 chuỗi token. Không chỉ vậy macro trong Rust sử dụng DSL riêng, DSL này gần với ngôn ngữ Rust nhưng có sự mở rộng biến đổi để phù hợp với vai trò macro, do đó không thể sinh cây AST cho macro.

Để xử lý trường hợp trên, ta có thể thêm 1 bước tiền xử lý mã nguồn để giải macro như C/C++. Chúng ta sẽ sử dụng đến thư viện cargo-expand, thư viện này có tác dụng đưa đoạn code Rust macro mà lập trình viên nhìn thấy thành đoạn code Rust mà compiler nhìn thấy. Đoạn code sau khi được mở rộng thì sẽ có được các thông tin bị ẩn đi như prelude mặc định của Rust bao gồm các hàm, symbol được built-in trong ngôn ngữ mà người dùng không phải import thủ công, các macro sẽ được xử lý, bao gồm cả declarative và procedural macro. Đối với declarative macro, thì macro built-in của ngôn ngữ như $println!$, $vec!$ hay kể cả declarative macro do người dùng định nghĩa cũng sẽ được giải.

Tuy nhiên, việc xử lý macro trước khi cho vào cây AST sẽ làm cho mã nguồn bị biến đổi so với mã nguồn gốc, đồng thời tăng kích cỡ và độ lớn của mã nguồn. Việc thêm các thông tin ẩn mà lập trình viên không nhìn thấy có thể gây nhầm lẫn cho người đọc mã nguồn. Điều này cũng đồng nghĩa với việc việc sinh cây AST cho mã nguồn sau khi xử lý macro sẽ phức tạp hơn, việc này sẽ làm tăng thời gian xử lý và độ phức tạp

\subsection{Module}

Cơ chế module trong Rust tương ứng với namespace trong C++, package trong Java. Module chia nhỏ mã nguồn thành các phần nhỏ hơn, tổ chức và quản lý mã nguồn, giúp tái sử dụng mã nguồn, giúp tránh xung đột tên biến, hàm giữa các module khác nhau.
Một module trong Rust có thể là một file riêng biệt hoặc một phần của một file khác. Các module có thể được tổ chức thành một hệ thống phân cấp, với các module con được khai báo bên trong các module cha.
% Có một số file được coi là file đặc biệt trong cấu trúc mã nguồn trong rust.
% Ví dụ như file mod.rs, đây là file module root của thư mục chứa nó, tất cả các module con trong thư mục đó sẽ được import thông qua module root mod.rs.
% Ngoài ra còn có file main.rs, lib.rs để đánh dấu điểm đầu vào của chương trình và xác định xem dự án là 1 thư viện hay 1 ứng dụng.
Rust còn cung cấp cơ chế workspace, cho phép quản lý nhiều dự án nhỏ trong cùng 1 dự án lớn, mỗi dự án là 1 thư mục con trong thư mục workspace.
Để kiểm soát khả năng truy cập, Rust sử dụng cơ chế visibility. Mặc định các thành phần trong module là private, để làm cho chúng có thể truy cập được từ các module khác, ta sử dụng từ khóa pub.
Cơ chế path resolution dùng để định danh 1 thành phần cấu trúc từ module khác ta mà ta có thể import. Path resolution có thể là đường dẫn tuyệt đối hoặc tương đối. Ví dụ dùng từ khóa $self$ để chỉ tới module hiện tại, dùng từ khóa $super$ để chỉ tới module cha của module hiện tại, $crate$ để chỉ tới module root của dự án.
Hệ thống module phức tạp của Rust làm tăng đáng kể độ khó việc xử lý quan hệ giữa các module trong cây AST và phân tích ngữ cảnh.
Việc xử lý các khái niệm như module con, module gốc, visibility, import và path resolution đòi hỏi một cơ chế phân tích tinh vi, do vậy hiện tại công cụ chưa xử lý module

% \subsection{Path}

% \begin{itemize}
%     \item Path có thể là Absolute Path hoặc Relative Path tùy vào bối cảnh module hiện tại, path có thể chỉ tới đối tượng trong cùng 1 module hoặc khác module
%     \item Path trong Rust là đường dẫn đến 1 đối tượng nào đó được định nghĩa trong mã nguồn như struct, trait, static, const, function
%     \item Cây AST sử dụng thư viện $syn$, $syn$ có thể lấy được path của 1 đối tượng nhưng không biết được đối đượng đang trỏ tới là static, const, hay function. Do đó đang không phân biệt được đâu là path của static, const, hay function
% \end{itemize}

% \subsection{Type Argument match Type Parameter}

