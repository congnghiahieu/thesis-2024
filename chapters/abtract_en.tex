\begin{center}
\textbf{\large{Abstract}}
\end{center}

\addcontentsline{toc}{chapter}{Abstract}
\begin{small}

\textbf{Abstract}:
Rust has become a widely used programming language in recent years.
Thanks to its memory safety guarantees and superior performance, Rust has been adopted by many large projects as a successor to C/C++.
These projects demonstrate that Rust is not just a trend but a language with long-term potential.
As major projects transition to Rust, ensuring safe source code and avoiding security vulnerabilities becomes a challenge.
With its growing popularity and strong adoption, the demand for source code analysis tools is also increasing.
Rust has gained its popularity recently, so the development of source code analysis tools is still in its early stages.
Some studies have been conducted, showing certain practical results and potential.
However, these studies are still limited in terms of performance, applicability to real-world projects, and integration with other research.
This thesis develops a source code analysis tool for the Rust programming language with the aim of overcoming these limitations.
The thesis adopts a static analysis approach based on code property graphs.
This allows the solution to be applied to large codebases and be more cost-effective compared to other solutions.
The tool performs analysis at the source code level rather than lower levels to ensure no loss of information regarding memory safety mechanisms.
Built on the extensible and reusable architecture of Joern, the tool is compatible with many existing analysis tools and research.
With code property graphs as the output, automated querying and analysis can be performed for various purposes.
Additionally, code property graphs can be applied to machine learning and reinforcement learning tasks to detect security vulnerabilities in Rust code.

\vspace*{1cm}
\textbf{Keywords}: Source code analysis, static analysis, Rust programming language
\end{small}