\section{Joern}

\subsection{Công cụ Joern}

\textbf{NOTE:} Nhấn mạnh Joern hiện tại chưa hỗ trợ Rust, chủ yếu hỗ trợ cho C/C++, Java là cực mạnh

Joern là một nền tảng mạnh mẽ dành cho việc phân tích mã nguồn, bytecode và mã nhị phân. Công cụ này tạo ra các đồ thị thuộc tính mã nguồn (code property graphs), một cách biểu diễn đồ thị của mã giúp cho việc phân tích mã nguồn đa ngôn ngữ trở nên dễ dàng hơn. Các đồ thị thuộc tính mã nguồn được lưu trữ trong một cơ sở dữ liệu đồ thị tùy chỉnh, cho phép khai thác mã nguồn thông qua các truy vấn tìm kiếm được xây dựng trong một ngôn ngữ truy vấn đặc thù dựa trên Scala. Joern được phát triển với mục tiêu cung cấp một công cụ hữu ích cho việc khám phá lỗ hổng bảo mật và nghiên cứu phân tích chương trình tĩnh.

Joern hỗ trợ các nhà phát triển và nhà nghiên cứu trong việc tìm kiếm và xác định các điểm yếu tiềm ẩn trong mã nguồn, giúp nâng cao chất lượng và bảo mật của phần mềm. Bên cạnh đó, Joern còn có khả năng phân tích mã nguồn đa ngôn ngữ, giúp các nhóm phát triển có thể làm việc với nhiều ngôn ngữ lập trình khác nhau mà không gặp trở ngại về công cụ. Với khả năng truy vấn mạnh mẽ và linh hoạt, Joern đã trở thành một công cụ quan trọng trong việc phân tích mã nguồn và phát hiện các lỗ hổng bảo mật. Bạn có thể tìm hiểu thêm về Joern tại địa chỉ https://joern.io/.

Joern hỗ trợ nhiều ngôn ngữ lập trình khác nhau. Các ngôn ngữ được hỗ trợ bao gồm C/C++, Java, JavaScript, Python, x86/x64, JVM Bytecode, Kotlin, PHP, Go, Swift, Ruby và C\#. Điều này cho thấy Joern có khả năng phân tích mã nguồn đa ngôn ngữ, giúp các nhà phát triển và nhà nghiên cứu có thể làm việc với nhiều ngôn ngữ lập trình khác nhau mà không gặp trở ngại về công cụ.
Tuy nhiên, Joern chưa hỗ trợ cho ngôn ngữ Rust.

\subsection{Đặc tả đồ thị thuộc tính mã nguồn của Joern}

\textbf{NOTE:} Thêm ý đặc tả CPG Joern ban đầu được viết cho ngôn ngữ C, sau đó mở rộng cho các ngôn ngữ khác như Java, PHP, ... Tuy nhiên đa phần các ngôn ngữ này đều là ngôn ngữ có cú pháp C-like và chủ yếu được code dưới OOP. Do vậy đối với ngôn ngữ có sự đan xen của các tính năng Functional Programming, thì bản đặc tả này tỏ ra có sự hạn chế

Đồ thị thuộc tính mã nguồn (Code Property Graph - CPG) là một cấu trúc dữ liệu được thiết kế để khai thác các cơ sở mã nguồn lớn nhằm tìm kiếm các mẫu lập trình. Những mẫu này được hình thành trong một ngôn ngữ đặc thù (DSL) dựa trên Scala. CPG đóng vai trò như một biểu diễn chương trình trung gian duy nhất cho tất cả các ngôn ngữ được Joern và phiên bản thương mại của nó là Ocular hỗ trợ.

Đồ thị thuộc tính là một trừu tượng chung được hỗ trợ bởi nhiều cơ sở dữ liệu đồ thị đương đại như Neo4j, OrientDB và JanusGraph. Trên thực tế, các phiên bản cũ của Joern đã sử dụng các cơ sở dữ liệu đồ thị mục đích chung làm nơi lưu trữ và ngôn ngữ truy vấn đồ thị Gremlin. Tuy nhiên, khi những hạn chế của cách tiếp cận này trở nên rõ ràng theo thời gian, chúng tôi đã thay thế cả hệ thống lưu trữ và ngôn ngữ truy vấn bằng cơ sở dữ liệu đồ thị OverflowDB của riêng mình.

Qwiet AI (trước đây là ShiftLeft) đã mã nguồn mở việc triển khai đồ thị thuộc tính mã nguồn và đặc tả của nó.

Các thành phần cấu thành đồ thị thuộc tính mã nguồn bao gồm:
\begin{itemize}
  \item \textbf{Các nút và loại của chúng:} Các nút đại diện cho các thành phần của chương trình. Điều này bao gồm các cấu trúc ngôn ngữ cấp thấp như phương thức, biến, và cấu trúc điều khiển, cũng như các cấu trúc cấp cao hơn như điểm cuối HTTP hoặc các kết quả phân tích. Mỗi nút có một loại, loại này chỉ ra loại thành phần chương trình mà nút đó đại diện, ví dụ, một nút với loại METHOD đại diện cho một phương thức, trong khi một nút với loại LOCAL đại diện cho khai báo của một biến cục bộ.
  \item \textbf{Cạnh có nhãn:} Quan hệ giữa các thành phần chương trình được biểu diễn thông qua các cạnh giữa các nút tương ứng của chúng. Ví dụ, để biểu thị rằng một phương thức chứa một biến cục bộ, chúng ta có thể tạo một cạnh với nhãn CONTAINS từ nút của phương thức đến nút của biến cục bộ. Bằng cách sử dụng các cạnh có nhãn, chúng ta có thể biểu diễn nhiều loại quan hệ khác nhau trong cùng một đồ thị. Hơn nữa, các cạnh có hướng để biểu thị, ví dụ, rằng phương thức chứa biến cục bộ nhưng không phải ngược lại. Nhiều cạnh có thể tồn tại giữa cùng hai nút.
  \item \textbf{Cặp khóa-giá trị:} Các nút mang các cặp khóa-giá trị (thuộc tính), trong đó các khóa hợp lệ phụ thuộc vào loại nút. Ví dụ, một phương thức có ít nhất tên và chữ ký, trong khi một khai báo biến cục bộ có ít nhất tên và loại của biến được khai báo.
\end{itemize}

Tóm lại, đồ thị thuộc tính mã nguồn là các đồ thị có hướng, được gán nhãn cạnh, và chứa các thuộc tính, và chúng tôi khẳng định rằng mỗi nút mang ít nhất một thuộc tính chỉ ra loại của nó. Điều này giúp cho việc phân tích mã nguồn trở nên dễ dàng và hiệu quả hơn, đồng thời mở ra nhiều khả năng cho việc tìm kiếm và phát hiện các mẫu lập trình trong các cơ sở mã nguồn lớn.
