\section{Các nghiên cứu liên quan}
\subsection{Evomaster}
EvoMaster là một công cụ mã nguồn mở được biết đến là công cụ kiểm thử  đầu tiên sử dụng trí tuệ nhân tạo để tự động sinh các ca kiểm thử cho ứng dụng mạng. Công cụ có thể sinh các ca kiểm thử API cho REST, GraphQL và RPC. Hơn nữa, EvoMaster cung cấp hai chế độ kiểm thử: kiểm thử hộp trắng  hoặc kiểm thử hộp đen. Cả hai chế độ đều bắt đầu bằng việc xử lý đặc tả OpenAPI đầu vào để lấy thông tin cho quá trình kiểm thử.

Đối với kiểm thử hộp trắng, chế độ này yêu cầu quyền truy cập vào mã nguồn chương trình và cũng là chế độ mang lại hiệu quả vượt trội của Evomaster. Nó sử dụng giải thuật tiến hóa (theo mặc định là thuật toán MIO [8]) để tạo ra các ca kiểm thử với mục tiêu tối đa hóa độ phủ mã nguồn. Cụ thể, đối với mỗi nhánh chưa được phủ, công cụ phát triển các ca kiểm thử  bằng cách liên tục tạo ra ca kiểm thử mới đồng thời loại bỏ những ca kiểm thử ít khả quan nhất  để thực hiện kiểm thử trên nhánh đó lặp lại cho đến khi đạt đến một mốc giới hạn thời gian. Kĩ thuật trong chế độ này tạo các kiểm thử mới thông qua việc lấy mẫu (sampling) hoặc biến đổi từ các ca kiểm thử trước đó (mutation). Ngược lại, trong chế độ kiểm thử hộp đen, Evomaster tập trung vào kĩ thuật kiểm thử mờ, cung cấp đầu vào cho việc kiểm thử bằng cách sinh ngẫu nhiên các trường hợp không hợp lệ hoặc hiếm gặp. Tuy nhiên, kết quả kiểm thử hộp đen có thể không hiệu quả bằng kiểm thử hộp trắng do thiếu phân tích mã nguồn. Điểm chung của hai phương pháp này là đều xem xét việc tìm kiếm các lỗi 5xx bên phía máy chủ.
\subsection{RESTler}
RESTler là công cụ kiểm thử mờ REST API có trạng thái đầu tiên được phát triển \cite{atlidakis2019restler}. Trong ngữ cảnh này, thuật ngữ "trạng thái" đề cập đến khả năng khám phá các trạng thái của dịch vụ mà chỉ có thể truy cập được thông qua chuỗi các lời gọi API được gửi đến hệ thống.

Để đạt được điều này, RESTler tạo ra các chuỗi yêu cầu bằng cách suy luận các phụ thuộc  giữa các
hoạt động (operation) trong đặc tả API và bằng cách phân tích các phản hồi được ghi lại trong các lần thực hiện các ca kiểm thử trước đó để tạo các ca kiểm thử mới mới. Tuy nhiên, vì RESTler chỉ tập trung vào việc sinh các chuỗi, các giá trị đầu vào của công cụ có thể không bao quát được nhiều các tình huống kiểm thử khả thi hoặc các trường hợp ngoại lệ, dẫn đến một số khía cạnh của API không được kiểm thử. Ngoài ra, mặc dù đã phân tích các phản hồi, nhưng nó không sử dụng thông tin này để tạo các ca kiểm thử nhằm ưu tiên tìm ra lỗi. Những hạn chế này khiến bộ kiểm thử của nó không đạt được độ phủ đầu ra và độ phủ mã nguồn mong muốn.
\subsection{Morest}

Morest là một công cụ kiểm thử API có trạng thái, dựa trên việc tạo ra Đồ thị Thuộc tính Dịch vụ RESTful (RPG) với các phụ thuộc được trích xuất giữa các API \cite{liu2022morest}. Đồ thị RPG có thể mô tả chi tiết các phụ thuộc của API và cho phép tinh chỉnh các phụ thuộc đã được ghi lại được một cách linh hoạt trong quá trình kiểm thử. Trong RPG, mỗi nút biểu thị một hoạt động hoặc một lược đồ, và cạnh biểu thị các phụ thuộc dữ liệu giữa chúng. Morest tạo ra các chuỗi các lời gọi API từ RPG trong quá trình kiểu thử. Bằng cách duyệt qua RPG theo phương pháp depth-first (duyệt theo chiều sâu), Morest có thể tạo các trường hợp kiểm thử cho các kịch bản sử dụng khác nhau. Tương tự như RESTler, Morest không tính đến tính đa dạng của đầu vào và đầu ra. Bên cạnh đó, các điều chỉnh động mà nó thực hiện dựa trên RPG chỉ có thể tìm ra các lỗi do sự kết hợp giữa các hoạt động, trong khi các tham số cũng có thể được kiểm thử theo những cách khác nhau.
\subsection{ARAT-RL}

ARAT-RL là một kỹ thuật kiểm thử API được phát triển mới đây, tận dụng học tăng cường để tạo ra các trường hợp kiểm thử một cách thích nghi \cite{kim2023adaptive}. So với các phương pháp khác, ARAT-RL sáng tạo ở chỗ, đã xác định trọng số của các tham số và hoạt động (operation), đồng thời điều chỉnh các trọng số này dựa trên thông tin nhận được từ phản hồi xuyên suốt quá trình kiểm thử. Dựa trên những trọng số đó, công cụ sẽ lựa chọn thứ tự kiểm thử các hoạt động (operation) và tham số, nhằm nâng cao hiệu suốt kiểm thử. Tuy nhiên, việc phân tích thông tin trả về trong quá trình kiếm thử để đánh giá các trọng số chưa toàn diện, ARAT-RL chỉ sử dụng thông tin về mã trạng thái của phản hồi, bỏ qua các thông tin hữu ích khác, dẫn đến việc chưa tối ưu việc đạt được độ phủ đầu ra. Mặt khác, để tạo dữ liệu kiểm thử, công cụ này tích hợp nhiều nguồn sinh dữ liệu khác nhau, giúp nâng cao độ đa dạng của dữ liệu đầu vào. Việc cài đặt các nguồn dữ liệu song song cho phép mở rộng quy mô theo chiều ngang bằng cách thêm các nguồn mới. Tuy nhiên, một số cài đặt nguồn hiện vẫn còn sơ sài và cần được cải thiện để nâng cao hiệu quả.