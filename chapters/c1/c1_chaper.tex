\chapter{Đặt vấn đề}
\label{chap:introduction}

Hiện nay, cùng với sự phát triển của công nghệ và độ phủ sóng của internet, ngành
công nghiệp phần mềm đang trải qua giai đoạn phát triển mạnh mẽ và đa dạng hóa không
ngừng, đi kèm với đó là hàng tỷ người sử dụng. Các hệ thống phần mềm phải luôn đảm
bảo tính sẵn sàng và ổn định bởi vì chỉ một sai xót xảy ra cũng sẽ gây ra những hậu
quả khôn lường. Trong quá trình xây dựng phần mềm, các đội ngũ phát triển phần mềm
thường dùng là sử dụng các công cụ phân tích mã nguồn cho việc
phát triển, kiểm thử và đảm bảo chất lượng phần mềm. Các công cụ phân tích mã nguồn
sẽ giúp phát hiện các lỗi lập trình, vấn đề về bảo mật hay việc sử dụng tài nguyên kém
hiệu quả mà không cần phải thực thi chương trình. Đồng thời, các công cụ này sẽ giúp
cải thiện chất lượng mã nguồn, đảm bảo lập trình viên sẽ tuân thủ các quy tắc và tiêu
chuẩn khi viết mã nguồn. Điều này sẽ giúp giảm thiểu thời gian và công sức cho quá
trình kiểm thử và đánh giá mã nguồn.

Hiện tại, trên thị trường đã có nhiều công cụ cung cấp khả năng phân tích mã nguồn
như SonarQube, ReSharper hay CodeClimate. Những công cụ này cung cấp khả năng
phân tích mã nguồn cho nhiều ngôn ngữ lập trình khác nhau như C/C++, Java, Javascript,
C... Bên cạnh đó, mỗi ngôn ngữ lập trình lại có thêm nhiều công cụ phân tích mã nguồn
khác nhau, nếu xét riêng cho ngôn ngữ lập trình Go, ta có một số công cụ tiêu biểu như
gosec1 , staticcheck2 hay govulncheck3 . Đây là các công cụ cung cấp khả năng phân tích
cú pháp và tìm lỗ hổng trong mã nguồn. Điểm chung của các công cụ này là đều sử dụng dạng cây cú pháp trừu tượng được định nghĩa sẵn của Go nên chúng chỉ dừng lại ở tìm
các lỗi dựa trên cú pháp mã nguồn mà không khai thác sâu hơn về luồng điều khiển hay
luồng dữ liệu của mã nguồn. Do vậy, những công cụ này thường cung cấp nhiều cảnh
báo giả và bỏ qua nhưng lỗi nghiêm trọng khi phân tích mã nguồn.

Khác với các công cụ vừa nêu, trên thị trường hiện nay còn xuất hiện một công cụ
là Joern. Joern là một nền tảng mã nguồn mở cung cấp khả năng phân tích mã nguồn,
mã bytecode và mã nhị phân. Một công cụ phân tích mã nguồn sẽ gồm ba thành phần
chính: phân tích cú pháp mã nguồn (parser), biểu diễn cấu trúc của mã nguồn và phân
tích cấu trúc biểu diễn đó [3]. Joern đóng vai trò là một công cụ giúp xử lý hai bước đầu
của quá trình phân tích mã nguồn. Thay vì chỉ sử dụng cây cú pháp trừu tượng, Joern
biểu diễn mã nguồn dưới dạng đồ thị thuộc tính mã nguồn và cung cấp chức năng truy
vấn khai thác đồ thị này bằng các câu lệnh truy vấn viết bằng ngôn ngữ Scala. Joern
được xây dựng với mục tiêu cung cấp chức năng tìm kiếm lỗ hổng trong mã nguồn và
cung cấp các công cụ để xây dựng các trình phân tích tĩnh sử dụng dữ liệu phân tích mà
Joern cung cấp. Điểm đặc biệt của Joern là nó cung cấp dạng đồ thị biểu diễn duy nhất
cho tất cả các ngôn ngữ mà nó hỗ trợ (bao gồm C, C++, Java, Javascript, Python...), điều
này đem đến khả năng phân tích đa ngôn ngữ, giúp giảm thiểu việc phải xử lý riêng cho
từng ngôn ngữ khi chúng ta tìm kiếm lỗi trong mã nguồn hay khi xây dựng các trình
phân tích tĩnh dựa trên Joern.

Mặc dù Joern đã cung cấp khả năng phân tích mã nguồn đối với ngôn ngữ Go tuy
nhiên vẫn còn rất hạn chế do chưa xử lý hết toàn bộ các thành phần mã nguồn có trong
Go và thiếu các thông tin về kiểu dữ liệu. Vì vậy khóa luận này xây dựng một công cụ
phân tích mã nguồn dành cho ngôn ngữ lập trình Go dựa trên mã nguồn của Joern. Công
cụ này sẽ xử lý các thành mã nguồn mà Go cung cấp một cách chi tiết và đầy đủ hơn,
đồng thời cũng xử lý triệt để kiểu dữ liệu để cung cấp đồ thị mã nguồn chi tiết nhất có
thể.

Khóa luận sẽ được trình bày theo cấu trúc như sau. Đầu tiên, Chương 2 thảo luận một số kiến thức cơ sở liên quan đến phân tích mã nguồn, cụ thể cho ngôn ngữ lập trình Rust. Chương 3  trình bày chi tiết quy trình xây dụng công cụ phân tích. Tiếp theo, Chương 4 sẽ trình bày kiến trúc, cài đặt công cụ và một số kết quả thực nghiệm đánh giá. Cuối cùng, tóm tắt những kết quả và kết luận sau quá trình phát triển công cụ sẽ được trình bày ở Chương 5.

