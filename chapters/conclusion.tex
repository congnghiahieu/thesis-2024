\chapter*{Kết luận}
\addcontentsline{toc}{chapter}{Kết luận}

Phân tích mã nguồn là một bước quan trọng trong quy trình đảm bảo chất lượng phần mềm, giúp phát hiện lỗ hổng bảo mật trong mã nguồn.
Việc phân tích mã nguồn với ngôn ngữ Rust trở nên khó khăn hơn do cú pháp đan xen lập trình hướng đối tượng và lập trình hướng hàm, hệ thống kiểu dữ liệu phức tạp và có các cơ chế đảm bảo an toàn bộ nhớ tạo ra sự khác biệt của Rust so với các ngôn ngữ đã xuất hiện trước đó.
Cũng đã có các nghiên cứu được đề xuất để giải quyết vấn đề này, nhưng các công cụ vẫn còn tồn tại những hạn chế về hiệu suất, khả năng áp dụng vào dự án thực tế hay khả năng áp dụng với các nghiên cứu khác.

Khóa luận phát triển một công cụ phân tích mã nguồn cho ngôn ngữ lập trình Rust với mục tiêu khắc phục được các điểm yếu kế trên.
Khóa luận lựa chọn hướng tiếp cận phân tích tĩnh thay vì các giải pháp kiểm chứng, kiểm thử động hay thực thi tượng trưng.
Điều này giúp giải pháp trong khóa luận có thể áp dụng vào các dự án mã nguồn lớn và tốn ít chi phí hơn so với các giải pháp còn lại.
Công cụ thực hiện phân tích ngôn ngữ Rust ở mức độ mã nguồn thay vì các cấp độ thấp hơn để đảm bảo không bị mất mát thông tin về các cơ chế đảm bảo an toàn bộ nhớ.
Công cụ lựa chọn đồ thị thuộc tính mã nguồn làm kiểu biểu diễn trung gian.
Việc này làm cho công cụ có thể tương thích, mở rộng với nhiều nghiên cứu đã có trước đó về đồ thị thuộc tính mã nguồn, đồng thời tái sử dụng được nhiều công cụ phân tích khác.

Công cụ được phát triển dựa trên Joern, kế thừa kiến trúc mở rộng và tái sử dụng mạnh mẽ.
Công cụ đã được kiểm thử trên các dự án mã nguồn mở lớn, thử nghiệm trên các đoạn mã nguồn có lỗ hổng trong thực tế và cho thấy tiềm năng khai thác.
Mặc dù công cụ đã đạt được kết quả nhất định, các hạn chế vẫn còn tồn tại.
Hiện tại công cụ chỉ có thể phân tích từng tệp mã nguồn Rust riêng lẻ, chưa xử lý được liên kết bằng module giữa các tệp mã nguồn.
Ngoài ra các macro trong Rust vẫn là khó khăn lớn do tính chất không được xử lý trước khi xây dựng cây AST.
Marco vẫn đang được giữ ở dạng chuỗi token và chưa thể xây dựng được cây AST từ đó.
Nhìn chung, bên cạnh các hạn chế liệt kê ở trên, phương pháp có tiềm năng khi áp dụng phân tích mã nguồn cho dự án Rust thực tế.
Phương pháp giải quyết được những hạn chế gặp phải của các nghiên cứu phân tích mã nguồn Rust trước đó.
