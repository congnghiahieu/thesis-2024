\chapter*{Kết luận}
\addcontentsline{toc}{chapter}{Kết luận}



Kiểm thử tự động API là một chủ đề nhận được sự quan tâm đặc biệt trong lĩnh vực nghiên cứu hiện nay, bởi vì API đóng vai trò quan trọng trong việc kết nối các hệ thống và ứng dụng. Với sự phát triển nhanh chóng của các công nghệ và nhu cầu tích hợp liên tục, việc đảm bảo chất lượng và hiệu suất của các API trở nên cấp thiết. Kiểm thử tự động API không chỉ giúp phát hiện và sửa chữa lỗi nhanh chóng mà còn nâng cao hiệu quả và độ tin cậy của các hệ thống phần mềm.

Khóa luận này đã trình bày một phương pháp đánh giá điểm thưởng tập trung vào việc tối ưu hóa mức độ phủ của bài toán kiểm thử API, đồng thời đảm bảo tính hiệu quả và toàn diện trong quá trình đánh giá. Phương pháp đề xuất đã xét đến nhiều khía cạnh quan trọng của API và quá trình kiểm thử, đồng thời mở rộng và tinh chỉnh nguồn sinh dữ liệu mới cho công cụ, dựa trên việc khai thác thông tin thu thập được trong quá trình kiểm thử.

Qua quá trình thực nghiệm trên một bộ dịch vụ kiểm thử uy tín, kết quả đạt được cho thấy phương pháp đề xuất có hiệu suất tốt hơn so với phiên bản gốc, nâng cao hiệu quả kiểm thử của công cụ. Cụ thể, công cụ được phát triển có thể xác định được 126,9 lỗi riêng biệt với mã trả về dạng \texttt{500}, vượt trội hơn 14 lỗi so với ARAT-RL. Về tốc độ, tại đa số thời điểm, công cụ được trình bày đều cho thấy hiệu suất tốt hơn so ARAT-RL. Đặc biệt, tại một số thời điểm, trên cùng một mốc yêu cầu, công cụ này vượt trội hơn với 15 lỗi tìm được nhiều hơn. Những kết quả này chứng minh rằng công cụ không chỉ cải thiện về mặt số lượng lỗi phát hiện được mà còn tăng tốc độ phát hiện lỗi, giúp tiết kiệm tài nguyên và nâng cao chất lượng phần mềm.

Trong tương lai, hướng phát triển của khóa luận này sẽ tập trung vào một số hướng đi quan trọng nhằm tiếp tục nâng cao hiệu quả và khả năng áp dụng của công cụ kiểm thử tự động API. Trước hết, cần tinh chỉnh và tối ưu hóa công cụ bằng cách cải thiện các thuật toán học tăng cường, tích hợp các thuật toán tiên tiến hơn để tăng cường khả năng học hỏi và thích ứng của công cụ trong các tình huống kiểm thử khác nhau. Đặc biệt, sử dụng Deep Q-learning, một phương pháp mạnh mẽ trong học tăng cường sâu, có thể mang lại hiệu quả cao. Deep Q-learning kết hợp giữa Q-learning và mạng nơ-ron sâu (DNN) để tạo ra một công cụ có khả năng xử lý các trạng thái phức tạp và không gian hành động lớn hơn. Mở rộng phạm vi áp dụng cũng là một mục tiêu quan trọng, với việc hỗ trợ nhiều loại API khác nhau, bao gồm RESTful API, GraphQL, và WebSocket, nhằm đảm bảo tính đa dụng của công cụ trong nhiều bối cảnh ứng dụng khác nhau, đồng thời tích hợp công cụ với các nền tảng và công cụ kiểm thử khác để tạo ra một môi trường kiểm thử toàn diện, phù hợp với nhiều loại dự án và yêu cầu khác nhau\cite{graphql,websocket}.

Việc kết hợp với các phương pháp tiên tiến như học máy và phân tích dữ liệu lớn sẽ giúp tự động nhận diện các mẫu lỗi phổ biến và dự đoán các vùng có khả năng xuất hiện lỗi cao trong API. Tích hợp công cụ kiểm thử tự động API vào các quy trình DevOps để thực hiện kiểm thử liên tục (CI/CD) sẽ đảm bảo rằng các lỗi được phát hiện và khắc phục kịp thời trong suốt quá trình phát triển phần mềm. Tăng cường khả năng mở rộng và tùy chỉnh của công cụ cũng là một bước đi cần thiết, bao gồm phát triển một giao diện người dùng trực quan và thân thiện, giúp người dùng dễ dàng cấu hình và sử dụng công cụ mà không cần nhiều kiến thức chuyên môn về kiểm thử, và kết nối công cụ với các hệ thống quản lý lỗi phổ biến để tự động báo cáo và theo dõi lỗi, giúp đội ngũ phát triển phần mềm dễ dàng quản lý và giải quyết các vấn đề phát sinh. Những cải tiến và phát triển này sẽ không chỉ nâng cao khả năng kiểm thử tự động API mà còn đóng góp vào sự phát triển của công nghệ kiểm thử phần mềm, đảm bảo rằng các hệ thống API hoạt động ổn định và đáng tin cậy, đáp ứng được các yêu cầu ngày càng cao của thị trường và người dùng.