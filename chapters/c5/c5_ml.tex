\section{Thực nghiệm học máy trên đồ thị CPG}

% Mục tiêu thực nghiệm là Bài toán phân loại
% Dữ liệu
% Độ đo đánh giá
% Baseline
% Kết quả thực nghiệm

Để chứng minh tiềm năng của đồ thị CPG dành cho ngôn ngữ Rust, khóa luận xây dựng một thực nghiệm sử dụng học máy để phát hiện lỗ hổng bảo mật mã nguồn Rust dựa trên đồ thị CPG.
Bài toán đặt ra là phân loại tệp mã nguồn Rust có lỗ hổng bảo mật và không có lỗ hổng bảo mật.
Thực nghiệm sẽ tiến hành so sánh độ hiệu quả giữa mô hình học máy sử dụng đồ thị CPG và mô hình học máy học sử dụng mã nguồn Rust truyền thống.
Mô hình học máy chỉ sử dụng dữ liệu mã nguồn Rust, tạm gọi là Baseline, sẽ sử dụng kĩ thuật Word2Vec để biểu diễn mã nguồn và Logistic Regression để phân loại.
Đối với bài toán học máy để phát hiện lỗ hổng bảo mật trên đồ thị CPG, trước đây đã có mô hình Devign, một mô hình kinh điển cho lớp bài toán này \cite{zhou2019devign}.
Devign đã được sử dụng cho ngôn ngữ C/C++ và cũng sử dụng Joern, do đó Devign hoàn toàn có thể sử dụng tương tự cho CPG của ngôn ngữ Rust trong khóa luận này.

Bộ dữ liệu được sử dụng bao gồm 260 tệp mã nguồn Rust, trong đó 130 tệp có lỗ hổng bảo mật và 130 tệp không có lỗ hổng bảo mật.
Đối với 130 tệp có lỗ hổng bảo mật, dữ liệu được trích xuất từ bài báo khoa học "A Closer Look at the Security Risks in the Rust Ecosystem" \cite{zheng2023closer}.
Còn 130 tệp không có lỗ bảo mật được thu thập từ 100 dự án mã nguồn mở của Rust nhiều sao nhất trên Github \cite{githubGithubRankingTop100RustmdMaster} và xác nhận thủ công.
Tập dữ liệu được chia thành 3 phần, 80\% dữ liệu được sử dụng cho quá trình huấn luyện, 10\% dữ liệu được sử dụng cho kiểm chứng, 10\% dữ liệu được sử dụng cho kiểm thử.

Với bài toán phân loại, thang đo đánh giá hiệu suất của mô hình sẽ bao gồm các chỉ số Accuracy, Precision, Recall và F1-score.
% ROC AUC, Precision-Recall AUC, MCC, Error Rate
Để đánh giá hiệu suất, mỗi mô hình được chạy 10 lần đối với tập dữ liệu như trên.
Kết quả của các chỉ số là giá trị trung bình của 10 lần chạy.
Kết quả thực nghiệm được thể hiện trong bảng \ref{table:c5_ml}.

\begin{table}[H]
    \centering
    \caption{Bảng so sánh mô hình Baseline và mô hình Devign}
    \label{table:c5_ml}
    \begin{tabular}{l @{\hskip 3cm} c @{\hskip 3cm} c}
        \hline
         & Baseline & \textbf{Devign (Rust)} \\
        \hline
        Accuracy & 0.73 & \textbf{0.77} \\
        Precision & 0.73 & \textbf{0.77} \\
        Recall & 0.70 & \textbf{0.99} \\
        F1-score & 0.72 & \textbf{0.87} \\
        % ROC AUC & 0.81 & \textbf{0.47} \\
        % Precision-Recall AUC & 0.74 & \textbf{0.78} \\
        % MCC & 0.46 & \textbf{0.01} \\
        % Error Rate & 0.27 & \textbf{0.23} \\
        \hline
    \end{tabular}
\end{table}

Trong bảng \ref{table:c5_ml}, mô hình Devign đã cho kết quả tốt hơn so với mô hình Baseline với hầu hết các chỉ số.
Ta thấy được Accuracy hơn 0.04, Precision hơn 0.04, Recall hơn 0.29, F1-Score hơn 0.15.
% , Precision-Recall AUC hơn 0.04 và Error Rate thấp hơn 0.04.
% Tuy nhiên, ROC AUC và MCC của mô hình Devign lại thấp hơn so với mô hình Baseline.
Điều này chứng tỏ mô hình Devign có tiềm năng áp dụng cho bài toán học máy phát hiện lỗ hổng bảo mật trên trên đồ thị CPG cho Rust.
Mã nguồn của thực nghiệm được lưu trữ tại địa chỉ \href{https://github.com/congnghiahieu/devign}{devign}.

Độ chính xác của mô hình Devign mới chỉ đạt được ở mức 0.77 bởi một vài nguyên do.
Thứ nhất là kích cỡ của bộ dữ liệu cho ngôn ngữ Rust, hiện tại các mã nguồn có lỗi được xác nhận đều lấy từ RUSTSEC Database.
Rust là ngôn ngữ mới phát triển gần đây, hệ sinh thái chưa lớn mạnh nên số lượng mã nguồn có lỗi được báo cáo không nhiều.
Thứ hai là đồ thị CPG của Rust vẫn chưa đầy đủ hoàn toàn.
Tồn tại các tính năng của Rust như marco, module chưa thể phân tích ra cây AST hay lấy các thông tin liên hệ giữa các nút.
% Các lớp thông tin về CFG, PDG vẫn chưa được hoàn thiện, hay các lớp thông tin khác có giá trị khai thác được sinh ra từ Joern cũng chưa được cung cấp.
Thứ ba là hạn chế cài đặt của mô hình Devign.
Phiên bản cài đặt của Devign là mô phỏng lại từ bài báo khoa học gốc.
Hiện tại Devign chỉ có thể sử dụng lớp thông tin về AST và CFG mà không có PDG, do vậy không thể khai thác được hết toàn bộ thông tin từ đồ thị CPG.
% chưa kể các lớp thông tin khác của Joern CPG.
Nếu có thể khắc phục được các hạn chế kể trên, mô hình Devign có thể đạt được kết quả tốt hơn nữa.
Dù tồn tại những hạn chế, kết quả thực nghiệm vẫn cho thấy tiềm năng khai thác to lớn của đồ thị CPG cho ngôn ngữ Rust.
Đồ thị CPG có thể được sử dụng cho các lớp bài toán cần đến phân tích mã nguồn như phát hiện lỗ hổng bảo mật, phân loại mã nguồn hay các ứng dụng khác trong tương lai.

