\section{Xây dựng cây cú pháp trừu tượng}

\textbf{NOTE:} Nhắc đến việc sử dụng thư viện Syn, Rust chưa có đặc tả chính thức mà có reference. Thư viện syn có ngữ pháp tuân theo reference này. Tất cả các loại AST sẽ tuân theo định nghĩa của thư viện Syn.

Quy trình xây dựng cây cú pháp trừu tượng bao gồm ba bước được thể hiện trong
Hình 3.2, chi tiết các bước được tiến hành như sau:

\begin{enumerate}
  \item Từ thư mục của dự án, lọc lấy các tệp mã nguồn Rust (các tệp có đuôi là .rs)
  \item Với mỗi tệp mã nguồn, sử dụng thư viện syn để phân tích thành cây cú pháp trừu tượng (AST) ứng với tệp đó.
  \item Chuyển đổi cây cú pháp trừu tượng thu được sang dạng cây tự định nghĩa. Về bản chất dạng cây tự định nghĩa này hoàn toàn giống với dạng cây gốc, tuy nhiên cây chỉ lưu các thông tin cần thiết cũng như bổ sung thêm một số thông tin để xác định nút cha, chữ ký nút và một số thông tin khác để phục vụ pha phân tích tiếp theo.
\end{enumerate}

Trong quá trình xây dựng cây cú pháp trừu tượng, các thành phần mã nguồn trong
Rust sẽ được phân tích thành dạng cây tương ứng. Dưới đây là một số hình ảnh ví dụ về
cây cú pháp trừu tượng ứng với các thành phần mã nguồn này.

Hình 3.3 biểu diễn một hàm được phân tích thành một nút định nghĩa hàm (Func-
tionDeclaration), nút này có các thuộc tính là chữ ký (signature), kiểu trả về (returnType)
và tên đầy đủ (fullName). Nút có các nút con gồm nút định danh (Identifier) biểu diễn
định danh của hàm, nút kiểu hàm (FunctionType) biểu diễn kiểu dữ liệu của hàm và nút
thân hàm (BlockStatement).

Ví dụ về biểu diễn một kiểu cấu trúc được thể hiện trong Hình 3.4. Nút TypeSpeci-
fication đại diện cho định nghĩa kiểu. Nút Identifier là nút định danh biểu diễn tên kiểu.
Nút StructType biểu diễn kiểu cấu trúc. Nút FieldList có các nút con là Field biểu diễn các thuộc tính của kiểu này.

\begin{table}[h]
\caption{Các nút trong cú pháp mã nguồn của Rust}
\label{table:nodeastrust}
  \begin{tabularx}{\textwidth}{|l|l|X|}
\hline
\textbf{STT} & \textbf{Tên nút} & \textbf{Mô tả} \\
\hline
1 & Comment & Nút biểu diễn một bình luận (comment) đơn bằng bằng cú pháp // hoặc /* */ \\
\hline
\end{tabularx}
\end{table}
