\section{Xây dựng đồ thị thuộc tính mã nguồn (CPG)}

Mục này sẽ trình bày quá trình công cụ phân tích ánh xạ cây cú pháp trừu tượng
thành cây đồ thị thuộc tính mã nguồn.

\subsection{Các loại đỉnh và cạnh của đồ thị thuộc tính mã nguồn CPG}

Đồ thị thuộc tính mã nguồn là dạng cấu trúc dữ liệu được kết hợp giữa cây cú pháp trừu tượng, đồ thị luồng điều khiển và đồ thị phụ thuộc chương trình.
Đồ thị này biểu diễn các đỉnh là các nút trong cây cú pháp trừu tượng và các cạnh của đồ thị biểu diễn mối quan hệ giữa chúng.
Đồ thị này giúp theo dõi các luồng điều khiển, các phụ thuộc trong mã nguồn.

Bảng \ref{table:c3_nodecpgjoern} dưới đây mô tả các đỉnh trong đồ thị thuộc tính mã nguồn

\footnotesize
\begin{longtable}{| p{.06\textwidth} | p{.31\textwidth} | p{.63\textwidth} |}
\hline
\textbf{STT} & \textbf{Tên nút} & \textbf{Mô tả} \\ \hline
1   & META\_DATA         & Nút này chứa siêu dữ liệu (metadata) của đồ thị. Một đồ thị CPG phải có và chỉ có duy nhất một nút META\_DATA.                                                   \\ \hline
2   & FILE              & Nút biểu diễn một tệp mã nguồn. Với mỗi tệp tin trong mã nguồn, đồ thị sẽ có chính xác một nút FILE để biểu diễn tệp mã nguồn đó.                               \\ \hline
3   & NAMESPACE         & Nút này biểu diễn một không gian tên (namespace) hoặc một gói (package), đối với Rust, nút này biểu diễn các gói trong mã nguồn.                                  \\ \hline
4   & NAMESPACE\_BLOCK   & Nút này biểu diễn một tham chiếu đến không gian tên (namespace) tương ứng. Nó chứa các khối mã nguồn được định nghĩa dưới không gian tên (namespace) tương ứng. \\ \hline
5   & METHOD            & Biểu diễn các hàm, phương thức hoặc biểu thức lambda.                                                                                                           \\ \hline
6   & PARAMETER\_IN      & Nút đại diện cho một tham số đầu vào của hàm.                                                                                                                   \\ \hline
7   & PARAMETER\_OUT     & Nút đại diện cho một tham số đầu ra của hàm.                                                                                                                    \\ \hline
8   & METHOD\_RETURN     & Nút này biểu diễn kiểu trả về của một hàm.                                                                                                                      \\ \hline
9   & MEMBER            & Biểu diễn thuộc tính của một kiểu cấu trúc (struct).                                                                                                            \\ \hline
10  & TYPE              & Nút đại diện cho một thể hiện (instance) của một kiểu dữ liệu.                                                                                                  \\ \hline
11  & TYPE\_ARGUMENT     & Biểu diễn đối số của một kiểu dữ liệu, có thể hình dung giống như các kiểu dữ liệu generic trong JAVA.                                                          \\ \hline
12  & TYPE\_DECL         & Biểu diễn một khai báo kiểu dữ liệu ví dụ như khai báo một kiểu cấu trúc (struct) hay khai báo một giao diện (interface).                                       \\ \hline
13  & BLOCK             & Biểu diễn một khối mã nguồn.                                                                                                                                    \\ \hline
14  & CALL              & Biểu diễn một lời gọi hàm.                                                                                                                                      \\ \hline
15  & CONTROL\_STRUCTURE & Biểu diễn một cấu trúc điều khiển ví dụ như câu lệnh if else, vòng lặp for, cấu trúc switch case.                                                               \\ \hline
16  & IDENTIFIER        & Nút định danh biểu diễn tham chiếu đến một biến.                                                                                                                \\ \hline
17  & JUMP\_LABEL        & Biểu diễn các cấu trúc nhảy như BREAK, CONTINUE, GOTO.                                                                                                          \\ \hline
18  & LITERAL           & Biểu diễn một hằng số như số nguyên hoặc chuỗi.                                                                                                                 \\ \hline
19  & LOCAL             & Biểu diễn một biến cục bộ.                                                                                                                                      \\ \hline
20  & METHOD\_REF        & Biểu diễn một tham chiếu đến hàm khi hàm đó được dùng làm tham số cho một lời gọi.                                                                              \\ \hline
21  & MODIFIER          & Biểu diễn một bộ chỉnh sửa (modifier) như static, public hay private.                                                                                           \\ \hline
22  & RETURN            & Biểu diễn một chỉ thị trả về (return instruction).                                                                                                              \\ \hline
23  & TYPE\_REF          & Biểu diễn tham chiếu đến một kiểu dữ liệu.                                                                                                                      \\ \hline
\caption{Các đỉnh trong đồ thị thuộc tính mã nguồn (CPG)}
\label{table:c3_nodecpgjoern}
\end{longtable}
\medskip

Bảng \ref{table:c3_edgecpgjoern} liệt kê các quan hệ trong đồ thị thuộc tính mã nguồn (CPG)

\footnotesize
\begin{longtable}{| p{.06\textwidth} | p{.31\textwidth} | p{.63\textwidth} |}
\hline
\textbf{STT} & \textbf{Tên nút} & \textbf{Mô tả} \\ \hline
1  & SOURCE\_FILE    & Quan hệ biểu diễn một nút là tệp gốc của một nút khác. Quan hệ cha con (biểu diễn quan hệ trong cây cú pháp mã nguồn (AST)).                                                                                 \\ \hline
2  & AST            & Cây cú pháp trừu tượng (Abstract Syntax Tree).                                                                                                                                                               \\ \hline
3  & CALL           & Quan hệ gọi, dùng để biểu diễn quan hệ giữa hàm/phương thức với nút gọi hàm/phương thức đó.                                                                                                                  \\ \hline
4  & CONTAINS       & Quan hệ bao gồm.                                                                                                                                                                                             \\ \hline
5  & CONDITION      & Quan hệ điều kiện, dùng để biểu diễn quan hệ giữa khối điều khiển như if else, switch hay for với biểu thức điều kiện của chúng.                                                                             \\ \hline
6  & ARGUMENT       & Quan hệ đối số kết nối các điểm gọi (kiểu nút CALL) với các đối số của chúng, cũng như các nút RETURN với các biểu thức trả về.                                                                              \\ \hline
7  & RECEIVER       & Quan hệ kết nối các điểm gọi (CALL) tới đối số nhận (receiver argument) của chúng. Một đối số nhận là một đối tượng mà phương thức đó thuộc về (có thể hiểu đối số nhận ở đây là con trỏ 'this' trong Java). \\ \hline
8  & CFG            & Quan hệ này biểu diễn luồng điều khiển từ nút nguồn đến nút đích.                                                                                                                                            \\ \hline
9  & DOMINATE       & Quan hệ biểu diễn tính kiểm soát (dominate) ngay lập tức từ nút nguồn đến nút đích.                                                                                                                          \\ \hline
10 & POST\_DOMINATE  & Quan hệ biểu diễn nút nguồn kiểm soát (dominate) ngay lập tức sau nút đích.                                                                                                                                  \\ \hline
11 & CDG            & Quan hệ biểu diễn nút đích phụ thuộc điều khiển vào nút nguồn.                                                                                                                                               \\ \hline
12 & REACHING\_DEF   & Quan hệ biểu diễn một biến được tạo ra từ nút nguồn và không bị gán lại giá trị trên đường đi tới nút đích.                                                                                                  \\ \hline
13 & CONTAINS       & Quan hệ này kết nối một nút tới hàm/phương thức chứa nó.                                                                                                                                                     \\ \hline
14 & EVAL\_TYPE      & Quan hệ kết nối một nút tới nút kiểu dữ liệu của nút đó.                                                                                                                                                     \\ \hline
15 & PARAMETER\_LINK & Quan hệ kết nối một nút tham số đầu vào với nút tham số đầu ra tương ứng của hàm/phương thức đó.                                                                                                             \\ \hline
\caption{Các cạnh trong đồ thị thuộc tính mã nguồn (CPG).}
\label{table:c3_edgecpgjoern}
\end{longtable}
\medskip

\subsection{Chuyển hóa cây cú pháp trừu trượng sang đồ thị thuộc tính mã nguồn}

Với các thông tin có được về cây cú pháp mã nguồn được lưu trong các tệp JSON,
đồ thị thuộc tính mã nguồn sẽ được dựng lên từ những thông tin đó.
Sau đây tôi sẽ mô tả việc ánh xạ các thành phần mã nguồn chính của Rust từ cây cú pháp mã nguồn sang đồ thị thuộc tính mã nguồn.

\textbf{Ánh xạ tệp (file) và gói (package)}

Hình 3.5 biểu diễn ánh xạ từ mã nguồn sang cây cú pháp trừu tượng và ánh xạ sang đồ thị thuộc tính mã nguồn.
Cây cú pháp trừu tượng của Rust không có nút biểu diễn gói, thông tin về gói sẽ được lưu trong nút tệp định nghĩa gói đó.
Khi ánh xạ nút tệp sang đồ thị thuộc tính mã nguồn sẽ biểu diễn được nút FILE và NAMESPACE tương ứng.
Nút NAMESPACE\_BLOCK biểu diễn thân của gói, nút này đóng vai trò làm nút cha của các định nghĩa bên trong thân của gói.
