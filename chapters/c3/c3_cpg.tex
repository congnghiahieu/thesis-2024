\section{Xây dựng đồ thị thuộc tính mã nguồn (CPG)}

\textbf{NOTE:} Nói về lý do lựa chọn sử dụng Joern (Joern Backend và Joern Frontend) để xây dựng đồ thị thuộc tính mã nguồn.

Mục này sẽ trình bày quá trình công cụ phân tích ánh xạ cây cú pháp trừu tượng
thành cây đồ thị thuộc tính mã nguồn.

\subsection{Joern Frontend và Joern Backend}

\subsection{Các loại đỉnh và cạnh của đồ thị thuộc tính mã nguồn CPG}

Đồ thị thuộc tính mã nguồn là dạng cấu trúc dữ liệu được kết hợp giữa cây cú pháp trừu tượng, đồ thị luồng điều khiển và đồ thị phụ thuộc chương trình. Đồ thị này biểu diễn các đỉnh là các nút trong cây cú pháp trừu tượng và các cạnh của đồ thị biểu diễn mối quan hệ giữa chúng. Đồ thị này giúp theo dõi các luồng điều khiển, các phụ thuộc trong mã nguồn.

\begin{table}[h]
\caption{Các đỉnh trong đồ thị thuộc tính mã nguồn (CPG)}
\label{table:nodecpgjoern}
  \begin{tabularx}{\textwidth}{|l|l|X|}
\hline
\textbf{STT} & \textbf{Tên nút} & \textbf{Mô tả} \\
\hline
1 & Comment & Nút biểu diễn một bình luận (comment) đơn bằng bằng cú pháp // hoặc /* */ \\
\hline
\end{tabularx}
\end{table}

\begin{table}[h]
\caption{Các cạnh trong đồ thị thuộc tính mã nguồn (CPG)}
\label{table:edgecpgjoern}
  \begin{tabularx}{\textwidth}{|l|l|X|}
\hline
\textbf{STT} & \textbf{Tên nút} & \textbf{Mô tả} \\
\hline
1 & Comment & Nút biểu diễn một bình luận (comment) đơn bằng bằng cú pháp // hoặc /* */ \\
\hline
\end{tabularx}
\end{table}

\subsection{Chuyển hóa cây cú pháp trừu trượng sang đồ thị thuộc tính mã nguồn}

Với các thông tin có được về cây cú pháp mã nguồn được lưu trong các tệp JSON,
đồ thị thuộc tính mã nguồn sẽ được dựng lên từ những thông tin đó. Sau đây tôi sẽ mô
tả việc ánh xạ các thành phần mã nguồn chính của Go từ cây cú pháp mã nguồn sang đồ
thị thuộc tính mã nguồn.

\textbf{Ánh xạ tệp (file) và gói (package)}

Hình 3.5 biểu diễn ánh xạ từ mã nguồn sang cây cú pháp trừu tượng và ánh xạ sang đồ thị thuộc tính mã nguồn. Cây cú pháp trừu tượng của Go không có nút biểu diễn gói, thông tin về gói sẽ được lưu trong nút tệp định nghĩa gói đó. Khi ánh xạ nút tệp sang đồ thị thuộc tính mã nguồn sẽ biểu diễn được nút FILE và NAMESPACE tương ứng.

Nút NAMESPACE\_BLOCK biểu diễn thân của gói, nút này đóng vai trò làm nút cha của các định nghĩa bên trong thân của gói.