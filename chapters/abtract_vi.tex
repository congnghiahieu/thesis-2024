\begin{center}
\textbf{\large{Tóm tắt}	}
\end{center}

\addcontentsline{toc}{chapter}{Tóm tắt}

\begin{small}
\textbf{Tóm tắt}:
Rust đang trở thành một ngôn ngữ lập trình vô cùng nổi bật và được sử dụng rộng rãi trong vài năm gần đây. Nhờ vào các tính năng nổi bật về đảm bảo an toàn bộ nhớ và hiệu suất vượt trội, mã nguồn của hàng ngàn dự án trên thế giới đã được viết lại hoặc viết mới bằng Rust, từ các hệ thống nhúng cho đến các ứng dụng web hiện đại. Rust đã được chọn làm ngôn ngữ phát triển của các dự án cấp độ doanh nghiệp như Servo - web engine của trình duyệt Mozilla, 1 phần của hệ điều hành Windows 11, và nhiều dự án khác.

Những dự án này cho thấy Rust không chỉ là một xu hướng mà còn là một ngôn ngữ có tiềm năng phát triển lâu dài. Với sự phổ biến của Rust, nhu cầu về các công cụ phân tích mã nguồn, kiểm thử và bảo mật cũng ngày càng tăng. Khi các dự án lớn chuyển sang sử dụng Rust, việc đảm bảo mã nguồn an toàn và không có lỗi trở thành một thách thức. Tuy nhiên hệ sinh thái và các công cụ cho phân tích mã nguồn Rust hiện tại vẫn chưa đáp ứng đủ. Rust chỉ mới được sử dụng phổ biến trong vài năm gần đây, vì vậy việc phát triển các công cụ phân tích vẫn còn trong giai đoạn đầu. Nhằm nâng cao độ ti n cậy của mã nguồn, các phương pháp kiểm thử phần mềm tự động đang nhận được đông đảo sự quan tâm và đang được áp dụng rộng rãi trong cả cộng đồng nghiên cứu lẫn các công ty phần mềm. Một trong những phương pháp sử dụng trong kiểm thử là phân tích mã nguồn hay phân tích tĩnh. Quá trình này là việc phân tích, đánh giá chất lượng mã nguồn và tìm ra các lỗi lập trình, lỗ hổng bảo mật mà không cần phải thực thi chương trình.

Khóa luận này trình bày phương pháp xây dựng công cụ phân tích mã nguồn dành cho ngôn ngữ Rust. Đầu vào của công cụ là các tệp mã nguồn Rust và đầu ra là đồ thị thuộc tính mã nguồn (CPG). Công cụ sử dụng đồ thị thuộc tính mã nguồn tuân theo chuẩn đặc tả và nền tảng có sẵn của Joern. Đầu ra có thể được lưu trữ trong cơ sở dữ liệu đồ thị, trực quan hóa, truy vấn, phân tích thủ công hoặc tự động nhằm phục vụ cho mục đích phân tích mã nguồn khác nhau. Ngoài ra CPG này có thể được áp dụng cho các kĩ thuật học máy để phát hiện lỗ hồng bảo mật như graph neural networks (GNN).

\vspace*{1cm}
\textbf{Từ khóa}: Phân tích mã nguồn, phân tích tĩnh, ngôn ngữ lập trình Rust

\end{small}