\begin{center}
\textbf{\large{Tóm tắt}	}
\end{center}

\addcontentsline{toc}{chapter}{Tóm tắt}

\begin{small}
Với sự phát triển mạnh mẽ của API, kiểm thử  API là một chủ đề được quan tâm đặc biệt trong lĩnh vực nghiên cứu hiện nay, bởi vì API đóng vai trò quan trọng trong việc kết nối các hệ thống và ứng dụng. Đi cùng sự phát triển nhanh chóng của công nghệ và nhu cầu tích hợp liên tục, việc đảm bảo chất lượng và hiệu suất của các API trở nên cấp thiết.

Khóa luận này đã trình bày công cụ kiểm thử tự động API ứng dụng học tăng cường với một phương pháp đánh giá điểm thưởng tập trung vào việc tối ưu hóa mức độ phủ trong kiểm thử API, đảm bảo tính hiệu quả và toàn diện trong quá trình đánh giá. 
Qua quá trình thực nghiệm trên một tập các dịch vụ kiểm thử  uy tín, kết quả cho thấy công cụ được phát triển có thể xác định được 126,9 lỗi riêng biệt với mã trả về dạng \texttt{500}, vượt trội hơn 14 lỗi so với ARAT-RL, công cụ kiểm thử tự động API mới nhất hiện nay \cite{kim2023adaptive}. Về tốc độ, tại đa số thời điểm, công cụ được trình bày đều cho thấy hiệu suất tốt hơn so ARAT-RL. Đặc biệt, tại một số thời điểm, trên cùng một mốc yêu cầu, công cụ này vượt trội hơn với 15 lỗi tìm được nhiều hơn. Những kết quả này chứng minh rằng công cụ không chỉ cải thiện về mặt số lượng lỗi phát hiện được mà còn tăng tốc độ phát hiện lỗi, giúp tiết kiệm tài nguyên và nâng cao chất lượng phần mềm.


\vspace*{1cm}
\textbf{Từ khóa}: Phân tích mã nguồn, phân tích tĩnh, ngôn ngữ lập trình Rust

\end{small}